\documentclass[aps,
12pt,
final,
oneside,
onecolumn,
musixtex, 
superscriptaddress,
centertags]{article}

\topmargin=-40pt
\textheight=650pt
\usepackage[english,russian]{babel}
\usepackage[utf8]{inputenc}

\usepackage{euscript}
\usepackage{supertabular}
\usepackage[pdftex]{graphicx}
\usepackage{amsthm,amssymb, amsmath}
\usepackage{textcomp}
\usepackage[noend]{algorithmic}
\usepackage[ruled]{algorithm}
\selectlanguage{russian}

\begin{document}
\begin{center}
% Title
\textbf{\Large Process Control} \\[1.0 cm]
\end{center}


Во время первой промышленной революции работа, проделанная человеческими мускулами, постепенно заменялась силой машин. Процесс контроля открыл дверь во вторую промышленную революцию, где рутинные функции человеческого разума и необходимость постоянного присутствия наблюдателей-людей также заботились о машинах. В результате оператор-оператор был освобожден от основной части утомительных и повторяющихся физических и умственных задач и смог сконцентрироваться на более творческих аспектах действующей отрасли. В этом смысле процесс управления сделал оптимизацию и, стало быть, начало третьей промышленной революции. В третьей промышленной революции традиционная цель максимизации количества продукции будет постепенно заменяться целью максимизации качества и долговечности выпускаемой продукции, минимизируя при этом потребление энергии и сырья и максимизируя рециркуляцию и повторное использование. Оптимизированное управление процессом будет двигателем этой третьей промышленной революции.

Возможно, автоматический контроль мог быть использован в Египте, Греции или Риме в связи с работами по орошению или водоснабжению, но мы этого не знаем точно. Первая известная автоматическая система управления, управляющая мухой,была установлена на паровом двигателе Уоттса более 200 лет назад в 1775 году.
Как показано на рис. 1.1a, шары мухи определяли скорость вращения вала и автоматически открывали подачу пара когда падение было зарегистрировано в скорости вращения. Еще один век прошел до того, как в 1868 году Джеймсом Кларком Максвелом был подготовлен первый математический анализ губернатора флайбола. 
Используя что-то, если он выполняет эту работу, прежде чем мы полностью поймем, почему и как она работает, является обычной практикой в управлении процессом. Янос фон Нейман сказал так: Нет смысла использовать точные методы, в которых нет ясности в концепциях и проблемах, к которым они должны применяться.

Расширение использования паровых котлов способствовало внедрению других автоматических систем управления, включая различные регуляторы давления пара и первые многоэлементные системы подачи воды котла. И здесь применение контроля процесса опережает его теорию, так как эта первая система управления форвардом была введена в то время, когда еще не был изобретен еще термин «форвард», не говоря уже о том, что он понимается. На самом деле первая общая теория автоматического управления, написанная Найквистом, не появлялась до 1932 года. В 1958 году был опубликован классический автоматический контроль процесса Дональда П. Экмана, и даже после этого в течение нескольких десятилетий большинство университетов и академических кругов в целом , рассматривал преподавание процесса управления, как будто это был еще один курс математики. Курсы по управлению процессами были сосредоточены на обучении методам анализа частотной области, в то время как инженеры по управлению технологическими процессами в этой области работали во временной области.

Только в конце 1960-х годов первое издание этого справочника, а также классические системы управления технологическими процессами Ф. Г. Шинского, стали признавать анализ управления процессом во временной области. Это соответствовало традициям этой профессии. На протяжении всей эволюции процесса контроля прогресс всегда делался практическими пользователями в полевых условиях, а не математически ориентированными теоретиками. Этот том и эта глава предоставляют ноу-хау, необходимые для современной практики управления технологическими процессами, но самый важный ингредиент может быть предоставлен только пользователем справочника: глубокое понимание его или ее собственный процесс. Прежде чем можно управлять процессом, нужно его полностью понять, и по этой причине это обсуждение начнется с описания «личностей» разных процессов.

Схематическое изображение флеш-губернатора. Когда вал изменяет скорость, вращающиеся флеш-шары перемещаются вверх и вниз. Затем рычаг управляет клапаном подачи пара для регулирования скорости пара. Если скорость вала слишком высокая, шары перемещаются вверх, поднимая сцепление, которое затем закрывает впускной клапан. Если скорость вала слишком низкая, шарики опускаются, опускают воротник и открывают впускной клапан.\\


ДИНАМИКА ПРОЦЕССА \\
  
Само собой разумеется, что пилоты океанского лайнера и сверхзвукового самолета должны иметь разные личности, потому что природа и характеристики их транспортных средств различны. Это справедливо и для управления промышленными процессами. Пилот (контроллер) должен быть правильно согласован с процессом, которым он управляет. Для этого разработчик процесса должен понимать «личность» процесса. Большинство процессов содержат резистивные, емкостные и мертвые элементы, которые будут определять их динамические и устойчивые ответы на расстройства. Прежде чем обсуждать управление процессами, в этом разделе представлены примеры вышеупомянутых компонентов, структурные блоки их личностей процессов.

Чтобы визуализировать поведение системы, необходимо предоставить визуальные описания компонентов личности этого процесса. Для этой цели используются блок-схемы (рисунок 1.1b). Двумя основными символами, используемыми на любой блок-схеме, являются круг и прямоугольная рамка. Круг представляет собой алгебраические функции, такие как сложение или вычитание. Круг всегда вводится двумя строками, но выходит из
только одна строка. Прямоугольная коробка всегда представляет собой динамическую функцию. В случае рис. 1.1b выходная или управляемая переменная (c) является функцией времени и также является функцией входной или управляемой переменной (m). Может появиться только одна строка, и только одна строка может оставить прямоугольный блок. «Системная функция», «личность» связанного компонента процесса, помещается внутри блока, а выход определяется из продукта системной функции и ввода.

Системная функция является символическим представлением о том, как этот компонент процесса меняет свой результат в ответ на изменение его ввода. Например, если системная функция является только усилением, внутри блока будет отображаться только константа (Kc, G или какой-либо другой символ). Коэффициент усиления процесса - это соотношение между изменением выходного сигнала (dc) и изменением входного сигнала, вызвавшего его (dm). Если вход (m) в блок является синусоидальным, выход (c) также будет синусоидальным. Коэффициент усиления процесса может быть результатом стабильного коэффициента усиления (Kp) и компонента динамического усиления (gp). Если коэффициент усиления изменяется с периодом входного (возбуждающего) синусоидального сигнала, он называется динамическим усилением (gp), а если он не влияет на этот период, он называется устойчивым усилением (Kp). Поэтому, если коэффициент усиления процесса можно разделить на стационарные и динамические компоненты, системная функция может быть задана внутри блока как (Kp) (gp). Динамический коэффициент усиления (gp) появляется как вектор, имеющий скалярную составляющую Gp и фазовый угол. Вскоре будет показано, что элементы процесса емкостного типа вводят фазовые сдвиги таким образом, что пиковая амплитуда входного синусоидального сигнала не вызывает одновременной амплитуды пика в выходном синусоидальном.\\

Процессы типа сопротивления \\

Падение давления через трубы и другое оборудование является наиболее очевидной иллюстрацией процесса сопротивления. Рисунок 1.1c иллюстрирует работу системы капиллярного потока, где поток линейно пропорционален падению давления. Этот процесс описывается установившимся усилением, равным сопротивлению (R). Поэтому, когда вход (m = поток) изменяется от нуля до m, выход (c = head) будет проходить мгновенное изменение шага от нуля до c = Rm. Ламинарное сопротивление потоку аналогично электрическому сопротивлению потоку тока. Единица сопротивления составляет sec / m2 в метрике и sec / ft2 в английской системе и может быть рассчитана с использованием уравнения 1.1 (1):

Когда поток турбулентный, сопротивление является функцией падения давления, а не квадратного корня падения давления, как в случае ламинарного (капиллярного) потока. Единицы сопротивления по-прежнему находятся в секундах на каждую область на английском или метрическом уровне и могут быть рассчитаны с использованием уравнения 1.1 (2):

Процесс потока жидкости обычно состоит из устройства для измерения расхода и регулирующего клапана последовательно, причем поток (с) проходит через оба. На рисунке l.ld показан такой процесс. Как видно из блок-схемы, процесс потока представляет собой процесс алгебраического и пропорционального (сопротивления). Управляемая переменная (m) - это открытие управляющего клапана, а управляемая переменная (c) - поток через систему. Изменение m приводит к немедленному и пропорциональному изменению c. Величина изменения зависит от коэффициента усиления процесса, также называемого чувствительностью процесса (Ka). Переменные нагрузки этого процесса представляют собой давление вверх и вниз по потоку (u0 и u2), которые являются независимыми, неконтролируемыми переменными. Изменение переменной нагрузки также приведет к немедленному и пропорциональному изменению контролируемой переменной (c = поток). Количество изменений зависит от их чувствительности к процессу или коэффициента усиления (Kb).
Общее уравнение процесса приведено в 1.1 (3):\\

Процессы емкостного типа \\

Большинство процессов включают некоторую форму емкости или возможности хранения. Эти емкостные элементы могут обеспечивать хранение материалов (газа, жидкости или твердых веществ) или хранения энергии (тепловой, химической и т. Д.). Тепловая емкость непосредственно аналогична электрической емкости и может быть рассчитана путем умножения массы объекта (W) на удельную теплоту материала, из которого он сделан (Cp). Единицами тепловой емкости являются BTU / ° F на английском или Cal / ° C в метрической системе.
Емкость емкости для хранения жидкости или газа может быть выражена в единицах площади (ft2 или в м2). Рис. 1 иллюстрирует эти процессы и дает соответствующие уравнения для расчета их емкостей. Емкость газа резервуара постоянна и аналогична электрической емкости. Емкость жидкости равна площади поперечного сечения резервуара на поверхности жидкости, и если площадь поперечного сечения постоянна, емкость также постоянна на любой головке.

Элемент чисто емкостного процесса может быть проиллюстрирован резервуаром, имеющим только приточное соединение (рис. L.lf). В таком процессе уровень (c) будет возрастать со скоростью, которая обратно пропорциональна емкости (площадь поперечного сечения резервуара) и через некоторое время будет затоплять бак. Уровень (с) в первоначально пустом резервуаре с постоянным притоком может быть определен путем умножения скорости притока (м) на период времени зарядки (t) и деления этого продукта на емкость резервуара (c = mt / C ). На рисунке l.lf показана система и описывается как системные уравнения, так и блок-схема емкостного элемента. При достижении системной функции операционная нотация дифференциала
используется уравнение, использующее замену s = d / dt. Таким образом, системная функция составляет 1 / Cs, а выход (c = head) получается путем умножения системной функции на вход (m).

Эффекты инерции относятся к второму закону Ньютона, регулирующему движение материи: \\
2F = (M X a) 1.1 (4) где \\
XF = чистая сила, действующая на массу \\
M = общая масса \\
a = ускорение этой массы \\
Эффекты инерции чаще всего связаны с механическими системами с движущимися компонентами, но они также важны в некоторых системах потоков, в которых жидкости должны быть ускорены или замедлены.\\

Сопротивление и емкость\\

Сопротивление и емкость, возможно, являются наиболее важными эффектами в промышленных процессах, связанных с переносом тепла, массопереносом и потоками жидкости. Те части процесса, которые имеют способность хранить энергию или массу, называются «мощностями», а те части, которые сопротивляются передаче энергии или массы, называются «сопротивлениями». Совокупный эффект обеспечения мощности через сопротивление - это время запаздывание, что очень важно для большинства динамических систем, обнаруженных в промышленных процессах. Рассмотрим, например, систему водонагревателя, показанную на рисунке l.lg, где могут быть легко определены емкости и сопротивления. Емкость - это способность резервуара и воды в резервуаре для хранения тепловой энергии. Вторая емкость - способность паровой катушки и ее содержимого хранить тепловую энергию. Сопротивление можно идентифицировать с передачей энергии от паровой катушки к воде из-за изолирующий эффект застойного слоя воды, окружающего катушку. Если мгновенное изменение скорости потока пара, температура горячей воды также изменится, но изменение не будет инст antaneous. Он будет вялым, требуя конечного периода времени для достижения нового равновесия. Поведение системы в течение этого переходного периода будет зависеть от количества материала, который должен быть нагрет в катушке и в резервуаре (определяется емкостью), и от скорости, с которой тепло может переноситься в воду (определяется сопротивление).\\

Постоянная времени процесса \\

Объединение емкостного типа (бака) с компонентом технологического процесса сопротивления (клапан) приводит к единому процессу постоянной времени. Если танк был первоначально пустым, а затем приток был начат с постоянной скоростью m, уровень в резервуаре повысился, как показано на рисунке l.lh, и в конечном итоге достигнет установившейся высоты c = Rm в баке ,
Чтобы развить общую системную функцию, необходимо объединить элемент емкости на рисунке l.lf с элементом сопротивления. Если входной переменной является приток (m), а выходная переменная - уровень (c), емкость резервуара равна разности между притоком (m) и оттоком (q):
C (dc / dt) = m - q 1,1 (5) В части сопротивления жидкости системы выход (c) равен произведению оттока (q) и сопротивления (R). Поскольку c = qR, поэтому q = c / R. \\
Подставляя c / R для q в уравнение 1.1 (5) и умножая обе стороны на R, получаем: \\
RC (dc / dt) + c = Rm 1.1 (6) Единица R - это время, деленное на площадь, а единица C - площадь; поэтому продукт RC имеет единицу времени. Это время (T) называется постоянной времени процесса. Экспериментально установлено, что после одной постоянной времени значение выходной переменной (с) одного процесса с постоянной по времени достигнет 63,2\% от ее конечного значения. Подставляя в Уравнение 1.1 (6) дифференциальный оператор s для d / dt и постоянную времени T для RC, и задним диапазоном уравнения в форму, где выход (c) равен входному (m), умноженному на системную функцию:
c = (R / (Ts + l)) m 1.1 (7) \\
Линейное дифференциальное уравнение первого порядка типично для большого класса компонентов и систем управления. \\
Общая форма такого уравнения равна T-c (t) + c (t) = Km (t) dt 1.1 (8) где T, K = константы процесса, постоянная времени и коэффициент усиления t = время \\
c (t) = выходной сигнал процесса \\
m (t) = ответ ввода процесса \\

Элементы процесса этого описания являются общими и обычно называются отставаниями первого порядка. Реакция системы первого порядка характеризуется двумя константами: постоянной времени T и коэффициентом усиления K. Усиление связано с усилением, связанным с процессом, и не влияет на время
характеристики ответа. Временные характеристики полностью связаны с постоянной времени. Постоянная времени является мерой времени, необходимого компоненту или системе для настройки на вход, и его можно охарактеризовать в терминах емкости и сопротивления (или проводимости) процесса:
T = сопротивление емкости емкостной емкости X
\\
Чтобы проиллюстрировать природу системы первого порядка, рассмотрим ответ, который возникает из ввода следующего вида:
\\
Решение дифференциального уравнения первого порядка для такого входа, учитывая начальное значение с, равное нулю, равно c (t) для
\\
Подробная информация о решении приведена в разделе 1.19.

В своих ответах (рис. 1, ч) две характеристики различают системы первого порядка: (1) Максимальная скорость изменения выхода происходит сразу же после ввода шага. (Заметим также, что если начальная скорость не изменилась, система достигнет конечного значения за период времени, равный постоянной времени системы.) (2) Фактический ответ получен, когда временной интервал равен постоянной времени из
система составляет 63,2\% от общего числа ответов. Эти две характеристики являются общими для всех процессов первого порядка.\\

Многократные постоянные процессы\\

На рисунке l.li показан процесс, в котором два резервуара соединены последовательно, и поэтому система имеет две постоянные времени, которые работают последовательно. Кривая отклика такого процесса медленнее, чем у одного процесса с постоянной по времени, показанного на рисунке l.lh, поскольку первоначальный отклик замедляется второй постоянной времени.
На рис. 1.1, j показаны ответы процессов, имеющих последовательно до шести временных констант. По мере увеличения числа постоянных времени кривые отклика становятся более тормозящими (коэффициент усиления процесса уменьшается), и общий отклик постепенно переходит в S-образную кривую реакции.\\

Ответ второго порядка\\
Из-за эффектов инерции и различных взаимодействий между сопротивлениями первого порядка и емкостными элементами некоторые процессы носят второстепенный характер и описываются следующим дифференциальным уравнением:

где \\
 = собственная частота системы \\
 = коэффициент демпфирования системы \\
 = системный выигрыш \\
 = время \\
 = входной сигнал системы \\
 = выходной отклик системы \\

Решение уравнения 1.1 (12) для ступенчатого изменения r (t) со всеми начальными условиями нуля может быть любым из семейства кривых, показанных на рисунке 1.1k.

В реальном решении необходимо учитывать три возможных случая, в зависимости от значения коэффициента демпфирования:
Когда 1.0, система, как говорят, недоукомплектована и будет превышать окончательное установившееся значение. Если 0,707, система будет не только превышать, но будет колебаться относительно конечного установившегося значения.
Когда 1.0, система, как говорят, завышена и не будет колебаться или превышать конечное положение в установившемся состоянии.
Когда 1.0, система считается критически затухающей и дает самый быстрый ответ без перерегулирования или колебаний.

Термин «естественной частоты» в уравнении второго порядка связан со скоростью ответа для определенного значения. Ответ, показанный на рис. 1.1k, нанесен на график с нормализованным временем, в течение которого фактическое время делится на собственную частоту, и поэтому большая частота имеет тенденцию сжимать ответ, а малая частота имеет тенденцию растягивать ответ. Собственная частота определяется в терминах «идеальной» или «без трения» ситуации, где 0.0. В такой ситуации ответ является устойчивой синусоидой с частотой колебаний, равной. Для случая, где не равен нулю, фактическая частота затухающего ответа связана с собственной частотой с помощью vr 1,1 (13) Процессы мертвого времени. Содействующий фактор динамики многих процессов, связанных с перемещением массы от одной точки до другой - отставание в перевозке, или мертвое время. Рассмотрим влияние трубопроводов на нагретую воду, чтобы достичь места на некотором расстоянии от нагревателя (рис. 1.11). Влияние изменения скорости пара на температуру воды на конце трубы будет не только зависеть от эффектов сопротивления и емкости в баке, но также будет зависеть от продолжительности времени, необходимого для транспортировки воды через труба. Все запаздывания, связанные с системой нагревателя, будут видны в конце трубы, но они будут отложены. Длина этой задержки называется транспортной задержкой (L) или мертвым временем. Величина определяется как расстояние, на которое транспортируется материал (1), деленное на скорость перемещения материала (v). В примере нагревателя L = v / 1 1.1 (14) Когда мертвое время присутствует в процессе, оно задерживает реакцию этого процесса. Если процесс представляет собой единый процесс с постоянной по времени, ответ, показанный на рисунке l.lh, будет задерживаться мертвым временем, как показано на рисунке 1.1m. Все остальные кривые отклика аналогично сдвинуты вправо на присутствие мертвого времени. Кривая отклика большинства процессов представляет собой S-образную кривую реакции, аналогичную одной из кривых на рисунке l.lj, и когда время мертвой точки присутствует, общий отклик принимает форму, показанную на рисунке l.ln.
Этот ответ типичен для большинства процессов в этой вселенной. В дополнение к промышленным процессам, экономические, культурные и биологические процессы также реагируют на изменение условий (нагрузки) таким образом. Этот процесс может быть водонагревателем или экономикой нашего глобального общества, и изменение может быть
температура входящей холодной воды или стоимость масла - в любом случае ответ на изменение будет схожим. Через некоторое время произойдет изменение, в течение которого измерения (температура горячей воды в одном случае, экономические показатели в другом) не будут затронуты. Это мертвое время процесса. Как только мертвое время (L) прошло, процесс начинает реагировать с его характерной скоростью, называемой чувствительностью процесса, и достигнет 63,2\% от его полного ответа за одну постоянную времени (T), если это один постоянный по времени процесс.

Мертвое время - худший враг хорошего контроля, поэтому инженер управления технологическими процессами должен сосредоточиться на его минимизации. Эффект мертвого времени можно сравнить с вождением автомобиля (процесс) с закрытыми глазами или с отключенным рулевым колесом в течение этого периода. Цель хорошей конструкции системы управления должна заключаться в том, чтобы свести к минимуму количество мертвого времени и свести к минимуму отношение мертвого времени к постоянной времени (L / T). Чем выше это отношение, тем меньше вероятность того, что система управления будет работать
и когда отношение L / T достигает 1,0 (L = T), управление традиционными ПИД (пропорционально-интегрально-производными) вряд ли будет работать. Различные способы сокращения мертвых время, как правило, связано с сокращением транспортных лаг. Это может быть достигнуто за счет увеличения скорости откачки или перемешивания, уменьшения расстояния между измерительным прибором и процессом, устранения систем отбора проб и тому подобного. Когда характер процесса таков, что отношение L / T должно превышать единицу, или если контролируемый процесс по своей сути является мертвым процессом (например, податчик ленты)
традиционное ПИД-управление заменяется управлением, основанным на периодических настройках, называемых типом управления выборкой и удержанием.\\

ПЕРЕМЕННЫЕ ПРОЦЕССЫ\\

Многие внешние и внутренние условия влияют на производительность технологического блока. Эти условия могут быть выражены в терминах переменных процесса, таких как температура, давление, расход, концентрация, вес, уровень и т. Д. Процесс обычно контролируется путем измерения одной из переменных, которые представляют состояние системы, а затем путем автоматической регулировки одной переменных, определяющих состояние системы. Как правило, переменная, выбранная для представления состояния системы, называется «управляемой переменной», а переменная, выбранная для управления состоянием системы, называется «управляемой переменной».\\
Управляемой переменной может быть любая переменная процесса, которая вызывает достаточно быструю реакцию и довольно легко манипулирует. Управляемая переменная должна быть переменной, которая наилучшим образом соответствует желаемому состоянию системы. Рассмотрим водоохладитель, показанный на рисунке l.lo. Целью кулера является поддержание подачи воды при постоянной температуре. Переменной, которая лучше всего представляет эту цель, является температура выходящей воды, вторая, и ее следует выбирать в качестве контролируемой переменной. В других случаях прямое управление переменной, наилучшим образом представляющей желаемое условие, невозможно. Рассмотрим химический реактор, показанный на рис. 1.1. Переменная, которая непосредственно связана с желаемым состоянием продукта, представляет собой состав потока продукта; однако в этом случае непосредственное измерение состава продукта не всегда возможно. Если композиция продукта не может быть измерена, используется другая переменная процесса, которая относится к композиции. Логичным выбором для этого химического реактора может быть поддержание постоянного давления и использование температуры реактора в качестве показателя состава. Такая схема часто используется при косвенном контроле состава.

\end{document}


